\section{Introduction}
Reinforcement Learning with Human Feedback (RLHF) enhances the learning process by incorporating human feedback into the reinforcement learning framework. In this chapter, we will implement a simple RLHF system using PyTorch. We will guide you through setting up the environment, defining the agent, integrating human feedback, and training the agent.

\section{Setup and Environment}
Before we start coding, ensure that you have PyTorch installed. You can install PyTorch via pip:

\begin{lstlisting}[language=bash]
pip install torch torchvision
\end{lstlisting}

Additionally, we will use other essential libraries such as numpy and gym.

\begin{lstlisting}[language=bash]
pip install numpy gym
\end{lstlisting}

\section{Defining the Environment}
We will use the OpenAI Gym environment for this example. Specifically, we'll use a simple environment like CartPole to demonstrate the RLHF implementation.

\begin{lstlisting}[language=python]
import gym
import numpy as np
import torch
import torch.nn as nn
import torch.optim as optim

env = gym.make('CartPole-v1')
\end{lstlisting}

\section{Defining the Agent}
Our agent will use a neural network to approximate the policy. We will define a simple feedforward neural network using PyTorch.

\begin{lstlisting}[language=python]
class PolicyNetwork(nn.Module):
    def __init__(self, state_dim, action_dim):
        super(PolicyNetwork, self).__init__()
        self.fc1 = nn.Linear(state_dim, 128)
        self.fc2 = nn.Linear(128, action_dim)
    
    def forward(self, x):
        x = torch.relu(self.fc1(x))
        x = torch.softmax(self.fc2(x), dim=-1)
        return x

state_dim = env.observation_space.shape[0]
action_dim = env.action_space.n

policy_net = PolicyNetwork(state_dim, action_dim)
optimizer = optim.Adam(policy_net.parameters(), lr=0.01)
\end{lstlisting}

\section{Collecting Human Feedback}
We will simulate human feedback for simplicity. In a real-world scenario, this feedback would come from a human supervisor.

\begin{lstlisting}[language=python]
def get_human_feedback(state):
    # Simulate human feedback by preferring actions that keep the pole balanced
    if state[2] > 0:  # Pole is falling to the right
        return 1  # Prefer moving the cart to the right
    else:  # Pole is falling to the left
        return 0  # Prefer moving the cart to the left

# Example usage:
state = env.reset()
action = get_human_feedback(state)
\end{lstlisting}

\section{Training the Agent with Human Feedback}
We will use the collected human feedback to train the policy network. The feedback will guide the agent towards the preferred actions.

\begin{lstlisting}[language=python]
def train_agent(num_episodes):
    for episode in range(num_episodes):
        state = env.reset()
        state = torch.tensor(state, dtype=torch.float32)
        done = False
        
        while not done:
            action_probs = policy_net(state)
            action = torch.argmax(action_probs).item()
            
            # Get human feedback
            preferred_action = get_human_feedback(state.numpy())
            
            # Compute loss based on human feedback
            loss = -torch.log(action_probs[preferred_action])
            
            optimizer.zero_grad()
            loss.backward()
            optimizer.step()
            
            state, reward, done, _ = env.step(preferred_action)
            state = torch.tensor(state, dtype=torch.float32)
            
            if done:
                break

# Train the agent for 100 episodes
train_agent(100)
\end{lstlisting}

\section{Evaluating the Agent}
After training, we evaluate the agent to see how well it performs.

\begin{lstlisting}[language=python]
def evaluate_agent(num_episodes):
    total_rewards = []
    
    for episode in range(num_episodes):
        state = env.reset()
        state = torch.tensor(state, dtype=torch.float32)
        done = False
        total_reward = 0
        
        while not done:
            action_probs = policy_net(state)
            action = torch.argmax(action_probs).item()
            
            state, reward, done, _ = env.step(action)
            state = torch.tensor(state, dtype=torch.float32)
            total_reward += reward
            
            if done:
                total_rewards.append(total_reward)
                break
    
    avg_reward = np.mean(total_rewards)
    print(f'Average Reward: {avg_reward}')

# Evaluate the agent for 10 episodes
evaluate_agent(10)
\end{lstlisting}

\section{Conclusion}
In this chapter, we implemented a simple RLHF system using PyTorch. We defined a policy network, simulated human feedback, and trained the agent using this feedback. This approach can be extended to more complex environments and real human feedback to create more sophisticated RLHF systems.

\bibliographystyle{plain}
\bibliography{references}